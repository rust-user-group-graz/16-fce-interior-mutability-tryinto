\documentclass{beamer}
\usetheme{metropolis}
%\setsansfont[BoldFont={Fira Sans SemiBold}]{Fira Sans Book}
%\setsansfont{Fontin}
%\setsansfont{Gillius ADF No2}
%\setsansfont{Phetsarath OT}
\setsansfont{Source Sans Pro}
\setmonofont{Source Code Pro}

\hypersetup{colorlinks=true,
            linkcolor=mRustLightOrange,
            menucolor=mRustLightOrange,
            pagecolor=mRustLightOrange,
            urlcolor=mRustLightOrange}
\usepackage{csquotes}
\usepackage{comment}
\usepackage{xcolor}
\usepackage{minted}

\newfontfamily\codefont{Source Code Pro}
\newcommand\code[1]{\,{\color[HTML]{884400}#1}\,}
\newcommand\source[1]{$\rightarrow$ via #1}

\title{Fluence Compute Engine (FCE), interior mutability, and TryInto}
\date{\today}
\author{Lukas Prokop}
\institute{RustGraz community\vfill\hfill\includegraphics[height=2cm]{images/rustacean-orig-noshadow.png}}
\begin{document}
\maketitle

%\section{Prologue}

\section{Fluence Compute Engine}

\begin{frame}[fragile]{FCE}
  What is FCE?\vspace{8pt}

  \begin{quote}
    \href{https://github.com/fluencelabs/fce}{Fluence Compute Engine (FCE)} is a general purpose Wasm runtime that could be used in different scenarios, especially in programs based on the ECS pattern or plugin architecture. It runs multi-module WebAssembly applications with interface-types and shared-nothing linking scheme.
  \end{quote}

  Presentation by \href{https://github.com/michaelvoronov}{michaelvoronov}:\vspace{8pt}
  \begin{quote}
    Mike Voronov has more than 10 years of experience in C++ and 2+ years experience in Rust and WebAssembly
  \end{quote}
\end{frame}

\section{Interior mutability}

\begin{frame}[fragile]{Reference semantics}
  \textbf{Revision:} shared versus mutable references.

  \begin{minted}{rust}
fn overwrite(base: &u32, new_value: &u32) {
  base = new_value;
}
fn main() {
  let value = 3;
  overwrite(&value, &42);
  println!("{}", value);
}
  \end{minted}
  \textbf{Does it compile?} \pause No, \emph{lifetime mismatch}.
\end{frame}

\begin{frame}[fragile]{Reference semantics}
  \begin{minted}{rust}
fn overwrite<'a>(base: &'a u32,
                 new_value: &'a u32) {
  base = new_value;
}
fn main() {
  let value = 3;
  overwrite(&value, &42);
  println!("{}", value);
}
  \end{minted}
  \textbf{Does it compile?} \pause No, \emph{cannot assign to immutable argument 'base'
}.
\end{frame}

\begin{frame}[fragile]{Reference semantics}
  \begin{minted}{rust}
fn overwrite<'a>(mut base: &'a u32,
                 new_value: &'a u32) {
  base = new_value;
}
fn main() {
  let value = 3;
  overwrite(&value, &42);
  println!("{}", value);
}
  \end{minted}
  \textbf{Does it compile?} \pause Yes, but prints \texttt{3}, not \texttt{42}.
\end{frame}

\begin{frame}[fragile]{Reference semantics}
  Modifying the reference \dots
  \begin{minted}{rust}
fn overwrite<'a>(mut base: &'a u32,
                 new_value: &'a u32);
  \end{minted}
  versus
  \begin{minted}{rust}
fn overwrite<'a>(base: &'a mut u32,
                 new_value: &'a u32);
  \end{minted}
  \dots{} modifying the value behind a reference.
\end{frame}

\begin{frame}[fragile]{Reference semantics}
  \begin{minted}{rust}
fn overwrite<'a>(base: &'a mut u32,
                 new_value: &'a u32) {
  *base = *new_value;
}
fn main() {
  let mut value = 3;
  overwrite(&mut value, &mut 42);
  println!("{}", value); // prints 42
}
  \end{minted}
\end{frame}

\begin{frame}[fragile]{Excursion: constant variables}
  \textbf{Small excursion:}
  Benedikt also pointed out the following snippet:
  \begin{minted}{rust}
fn overwrite<'a>(base: &'a mut u32,
                 new_value: &'a u32) {
  *base = *new_value;
}
fn main() {
  let mut value = 3;
  overwrite(&mut 42, &mut 42);
  println!("{}", value); // prints 3
}
  \end{minted}
  Apparently, rust creates \emph{two} local variables with value \emph{42}.
  This is interesting since constants of same value need not be allocated twice usually.
\end{frame}

\begin{frame}[fragile]{Mutability semantics}
  \begin{enumerate}
    \item A variable is mutable; or not.
    \item A reference is mutable; or not.
    \item If a struct instance bound to a variable is mutable; its members are mutable too (\emph{inherited mutability}).
    \item Mutability checks happen at \emph{compile time}.
  \end{enumerate}
\begin{minted}[fontsize=\small]{rust}
struct User {
  id: u32,
  posts_count: u32,
}
let mut meisterluk = User {
  id: 1,
  posts_count: 42
};
\end{minted}
\end{frame}

\begin{frame}[standout]
  What about a data structure that modifying elements in the background without knowledge of the user?
\end{frame}

\begin{frame}[fragile]{Interior mutability types in rust}
  Multithreaded? Use atomic datatypes or locks (Mutex, RwLock, \dots)!
  Otherwise, \dots

  \begin{description}
    \item[\mintinline{rust}{Cell<T>}] Provides interior mutability for some value.
    \item[\mintinline{rust}{RefCell<T>}] Provides interior mutability for some value and returns references in its API.
    \item[\mintinline{rust}{UnsafeCell<T>}] Underlying primitive for the types above
  \end{description}
  \dots{} implemented with \emph{runtime} checks!
\end{frame}

\begin{frame}[fragile]{Cell}
  \mintinline{rust}{Cell<T>:} This type wraps an existing value and provides interior mutability.

  \begin{minted}[fontsize=\scriptsize]{rust}
use std::cell::Cell;
struct User {
  id: u32,
  count_posts: Cell<u32>,
}

fn main() {
  let meisterluk = User {
    id: 1,
    count_posts: Cell::new(42),
  };
  // meisterluk.id = 4;  // error: meisterluk is not mutable
  meisterluk.count_posts.replace(55);
  println!("User=({}, {})",
    meisterluk.id,
    meisterluk.count_posts.get()
  );
}
  \end{minted}
\end{frame}

\begin{frame}[fragile]{Cell API}
  \mintinline{rust}{Cell<T>:} This type wraps an existing value and provides interior mutability.

  \begin{enumerate}
    \item \mintinline{rust}{pub const fn new(value: T) -> Cell<T>}
    \item \mintinline{rust}{pub fn set(&self, val: T)}
    \item \mintinline{rust}{pub fn swap(&self, other: &Cell<T>)}
    \item \mintinline{rust}{pub fn replace(&self, val: T) -> T}
    \item \mintinline{rust}{pub fn into_inner(self) -> T}
    \item \mintinline{rust}{pub const fn new(value: T) -> Cell<T>}
  \end{enumerate}
\end{frame}

\begin{frame}[fragile]{RefCell}
  \mintinline{rust}{RefCell<T>:} This type wraps an existing value and provides references for interior mutability.

  \begin{minted}[fontsize=\scriptsize]{rust}
use std::cell::RefCell;

struct User {
  id: u32,
  count_posts: RefCell<u32>,
}

fn main() {
  let meisterluk = User {
    id: 1,
    count_posts: RefCell::new(42),
  };
  meisterluk.count_posts.replace(55);
  println!("User=({}, {})",
           meisterluk.id,
           meisterluk.count_posts.borrow());
}
\end{minted}
\end{frame}

\begin{frame}[fragile]{RefCell API}
  \begin{enumerate}
    \item \mintinline{rust}{pub const fn new(value: T) -> RefCell<T>}
    \item \mintinline{rust}{pub fn borrow(&self) -> Ref<'_, T>}
    \item \mintinline{rust}{pub fn try_borrow(&self)} \\
          \mintinline{rust}{-> Result<Ref<'_, T>, BorrowError>}
    \item \mintinline{rust}{pub fn borrow_mut(&self) -> RefMut<'_, T>}
    \item \mintinline{rust}{pub fn try_borrow_mut(&self)} \\
          \mintinline{rust}{-> Result<RefMut<'_, T>, BorrowMutError>}
    \item \mintinline{rust}{pub fn as_ptr(&self) -> *mut T}
    \item \mintinline{rust}{pub fn get_mut(&mut self) -> &mut T}
    \item \mintinline{rust}{pub unsafe fn try_borrow_unguarded(&self)} \\
          \mintinline{rust}{-> Result<&T, BorrowError>}
  \end{enumerate}
\end{frame}

\begin{frame}[fragile]{RefCell}
  \begin{minted}[fontsize=\scriptsize]{rust}
use std::cell::RefCell;

struct User {
  id: u32,
  count_posts: RefCell<u32>,
}

fn main() {
  let meisterluk = User {
    id: 1,
    count_posts: RefCell::new(42),
  };
  let a = meisterluk.count_posts.borrow_mut();
  let _ = meisterluk.count_posts.borrow_mut();
}
  \end{minted}

  Runtime error!
  \begin{minted}[fontsize=\scriptsize]{text}
thread 'main' panicked at 'already borrowed: BorrowMutError',
      src/main.rs:14:36
note: run with `RUST_BACKTRACE=1` environment variable
      to display a backtrace    
  \end{minted}
\end{frame}

\section{TryInto / Into / TryFrom / From}

\begin{frame}[fragile]{Motivation}
  \begin{enumerate}
    \item Sometimes it is convenient to convert one type into another (\emph{coercion}, \emph{casting})
    \item Usually done explicitly in rust with \mintinline{rust}{as} keyword
    \item But when calling a function, we often know the source and target target. How can we convert it?
  \end{enumerate}
\end{frame}

\begin{frame}[fragile]{4 traits}
  \begin{minted}[fontsize=\small]{rust}
pub trait Into<T>: Sized {
  fn into(self) -> T;
}
pub trait TryInto<T>: Sized {
  type Error;
  fn try_into(self)
    -> Result<T, Self::Error>;
}
pub trait From<T>: Sized {
  fn from(_: T) -> Self;
}
pub trait TryFrom<T>: Sized {
  type Error;
  fn try_from(value: T)
    -> Result<Self, Self::Error>;
}
  \end{minted}
\end{frame}

\begin{frame}[fragile]{4 traits}
  \begin{enumerate}
    \item Try traits permit conversion to fail.
    \item All traits are reflexive (T can be converted \emph{into} T).
    \item Prefer to implement \mintinline{rust}{TryFrom} instead of \mintinline{rust}{TryInto}
    \item Implementing \mintinline{rust}{From} automatically provides one with an implementation of \mintinline{rust}{Into}.
  \end{enumerate}
\end{frame}

\begin{frame}[fragile]{Example implementation}
  Example via \href{https://doc.rust-lang.org/stable/rust-by-example/conversion/try_from_try_into.html}{Rust by Example}:
  \begin{minted}[fontsize=\small]{rust}
use std::convert::{TryFrom, TryInto};

#[derive(Debug, PartialEq)]
struct EvenNumber(i32);

impl TryFrom<i32> for EvenNumber {
  type Error = ();
  fn try_from(value: i32)
    -> Result<Self, Self::Error>
  {
    if value % 2 == 0 {
        Ok(EvenNumber(value))
    } else {
        Err(())
} } }
\end{minted}
\end{frame}

\begin{frame}[fragile]{Example implementation}
  \begin{minted}[fontsize=\small]{rust}
fn main() {
  // TryFrom
  assert_eq!(EvenNumber::try_from(8),
             Ok(EvenNumber(8)));
  assert_eq!(EvenNumber::try_from(5),
             Err(()));

  // TryInto
  let result: Result<EvenNumber, ()> = 8i32.try_into();
  assert_eq!(result, Ok(EvenNumber(8)));
  let result: Result<EvenNumber, ()> = 5i32.try_into();
  assert_eq!(result, Err(()));
}
  \end{minted}
\end{frame}

\begin{frame}[fragile]{Strings to vectors}
  Strings can be converted into a vector of bytes.

  \begin{minted}{rust}
fn main() {
    let a: Vec<u8> = String::from("hello").into();
    println!("{}", a[0]);
}
  \end{minted}
\end{frame}

\begin{frame}[fragile]{Into as trait bound}
  Trait bounds can specify that any type convertible into another will be accepted.
  \begin{minted}[fontsize=\small]{rust}
fn is_hello<T: Into<Vec<u8>>>(s: T) {
  let bytes = b"hello".to_vec();
  assert_eq!(bytes, s.into());
}

let s = "hello".to_string();
is_hello(s);
  \end{minted}
\end{frame}

\section{Epilogue}

\begin{frame}[fragile]{Quiz}
  \begin{description}
    \item[Who maintains the WASM standard?] \hfill{} \\
      ~\uncover<2->{WebAssembly became a World Wide Web Consortium recommendation in Dec 2019}
    \item[What is mutable about a mutable reference?] \hfill{} \\
      ~\uncover<3->{The value the reference is pointing to}
    \item[How can you implement casting for your own type?] \hfill{} \\
      ~\uncover<4->{Implement Into/TryInto or From/TryFrom}
    \item[When do you implement Into instead of TryInto?] \hfill{} \\
      ~\uncover<5->{If the conversion cannot fail under any circumstances.}
  \end{description}
\end{frame}

\begin{frame}[fragile]{Next time}
  \begin{tabular}{ll}
    \textbf{Next meetup}   & Wed, 2021/01/27 \\
    \textbf{Topic}         & Cross-compilation: \\
                           & compiling for the raspberry PI
  \end{tabular}
\end{frame}

\begin{frame}[standout]
  Thank you!

  \includegraphics[width=40pt]{images/rustacean-flat-happy.png}
\end{frame}

\end{document}
